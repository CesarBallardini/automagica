\documentclass[12pt,twoside,openright]{book}

\usepackage[a5paper,hmarginratio={3:2},bottom=0.8in,top=0.8in]{geometry}

\usepackage{times}
\usepackage[Lenny]{fncychap}

\usepackage[spanish]{babel}
\usepackage[utf8x]{inputenc}
\usepackage[T1]{fontenc}

\usepackage{mathptmx}
\usepackage{etoolbox}

\usepackage[titles]{tocloft}

\usepackage{pdfpages}

\renewcommand{\cftchapleader}{\cftdotfill{\cftdotsep}}

% change the space before the titles
\makeatletter
\patchcmd{\@makechapterhead}{\vspace*{50\p@}}{\vspace*{0pt}}{}{}
\patchcmd{\@makeschapterhead}{\vspace*{50\p@}}{\vspace*{0pt}}{}{}
\makeatother

% change the space after the titles
\renewcommand{\DOTI}[1]{%
    \raggedright
    \CTV\FmTi{#1}\par\nobreak
    \vskip 10pt}
\renewcommand{\DOTIS}[1]{%
    \raggedright
    \CTV\FmTi{#1}\par\nobreak
    \vskip 10pt}


\title{Capítulos sueltos}
\author{Juan José Conti}
\date{}



% Avoid widows and orphans
\widowpenalty=10000
\clubpenalty=10000

\begin{document}

\pagenumbering{gobble}

\pagestyle{plain}


% IF PRINT
%\includepdf{empty.pdf}
%\includepdf{empty.pdf}

\maketitle

\cleardoublepage

\thispagestyle{empty}
\noindent
Edición automágica, 2016.\\

\vspace{0.5cm}

\noindent
\emph{Capítulos sueltos} lleva la licencia
\emph{Creative Commons Attribution - NonCommercial - ShareAlike 4.0 Iternational}.
Esto significa que podés compartir esta obra y crear obras derivadas
mencionando al autor, pero no ha\-cer un uso comercial de ella.

\vfill

\noindent
%Más información sobre este libro:\\
\\

\cleardoublepage


\renewcommand*\contentsname{Índice}

\tableofcontents


\cleardoublepage

\pagenumbering{arabic}



\chapter*{Un bien más para el consumo} \addcontentsline{toc}{chapter}{Un bien más para el consumo}


Hoy lo escuché en las noticias.

Sé que en otros países ya se puede, pero pensé que me iba a morir antes de que lo permitieran acá.

De los viajes en el tiempo se sabe desde hace más de cinco años. El primer viaje publicitado fue en Estados Unidos, durante un programa de televisión. Los encargados de la demostración fueron dos militares. Uno se llamaba John y el otro Carl pero me voy a referir a ellos como Juan y Carlos, para que me sea más fácil contarles la anécdota.

Primero salió un paper en una revista científica y en menos de dos meses ya había grupos de investigación en todo el mundo trabajando (ahí fue cuando lo vi en un diario de aca). En algunos artículos, además de los enunciados teóricos que todos repiten pero casi nadie entiende, se describían experimentos. Recuerdo incluso el recuadrito con la formula esa: la copiaban y pegaban como si fuera un elemento de decoración. Todos los artículos del tema incluían la fórmula: ni siquiera la volvían a escribir, la copiaban y la pegaban como una imagen más, como si fuera una foto.

La de la fórmula podría ser una historia en sí misma: pasó a ser un elemento de cultura pop: una vez la vi escrita con aerosol en una pared.

Se estuvo hablando del tema como un año hasta que por fin se anunció una demostración pública. La propaganda empezó un mes antes, inundando todos los medios de comunicación. No podías ver la hora en tu reloj sin que te salte uno de esos cartelitos: “27 de febrero, primer demostración pública de viaje en el tiempo”. Era en el prime time norteamericano pero acá era de tarde. Un rato antes de la hora, me senté frente a la pantalla. En el centro del estudio, sentados en dos sillones, bien al estilo de los programas de entrevistas de los años dos mil, estaban Juan y Carlos. Los dos de verde musgo.

Entonces el conductor del programa, vestido en un traje azul cansino, entró con un micrófono de mano y saludó al público. El micrófono de mano era más bien un detalle de los ochenta, pero no desentonaba en esa puesta en escena estrafalaria. El hecho de que hubiera público en una emisión, también era totalmente anacrónico, pero supongo que la mezcla habrá logrado el objetivo que buscaban los productores de aquel evento.

“¡Muy buenas noches, América! ¡Muy buenas noches, mundo!”, gritó el conductor y una catarata de gritos y aplausos estremeció el lugar. “Estamos aquí con el sargento Juan y el cabo Carlos, quienes serán los encargados de poner en práctica un ejercicio de libro de viaje en el tiempo. Podría por favor, sargento Juan, contarnos en qué consiste el experimento. ¿Está bien que lo llame así, experimento?”.

Juan habló serio, con voz grave: “sí, cómo no, señor, le explico. En primer lugar una pequeña corrección. Tiene razón en dudar del término. Lo que vamos a realizar aquí no es un experimento ya que no es un procedimiento mediante el cual se trate de comprobar una hipótesis relacionada con un determinado fenómeno, sino que lo que vamos a hacer es ejecutar un ejercicio controlado en el cual tanto las acciones como los resultados son fijos. Lo que haremos aquí tendrá consecuencias que se conocen de antemano. Sus efectos han sido medidos y se ha determinado que su ejecución es segura. El viaje en el tiempo es una ciencia en desarrollo y debe ser utilizada con estricto cuidado y con rigurosa supervisión.”

“Entendido... sargento Juan”, y el conductor hizo una especie de venia ridícula. “Cuéntenos entonces, ahora sí, en qué consistirá la ejecución del ejercicio controlado” y mencionó las últimas cuatro palabras de un modo sutilmente diferente, como si no le hubiese gustado para nada la desacreditación en cámara. El sargento Juan no hizo caso y explicó:

“El ejercicio consiste en lo siguiente: esto que tengo aquí en mi muñeca, con forma de reloj, es un dispositivo construido en base a la teoría original más muchos desarrollos propios que ha hecho el gobierno…”

“La fórmula”, interrumpió el conductor, diciéndole con solemnidad esa palabra a la vez genérica y vacía.

“Sí”, sentenció Juan, mientras que con la mirada le decía que no quería ser interrumpido.

“Voy a viajar media hora al pasado y voy a encontrarlo al cabo Carlos solo en su camarín. Sabemos que él va a estar solo, nos hemos asegurado de eso. Si bien yo estoy preparado psicológicamente para encontrarme conmigo mismo, no hay razón para forzar ese momento incómodo. Apenas lo encuentre voy a raparle la cabeza y luego voy a regresar”.

Y entonces, apretó algo en su reloj y desapareció. El estudio pareció congelarse. El público se quedó en silencio. El conductor también, aunque tenía la boca abierta y el micrófono apuntando a ella.

Dos minutos después, Juan reapareció y siguió hablando: “Como se habrán dado cuenta, el ejercicio ya tuvo lugar”.

Vi en la pantalla cómo el público abucheaba. Yo también lo hice. Carlos había tenido la cabeza rapada desde el momento que se sentó en el sillón. Lo único que el sargento Juan había hecho era un truco de desaparición de prestidigitador de ferias.

El conductor, más confundido que enojado, hizo un gesto con las manos hacia el público para intentar calmarlos. Cuando el clamor bajó, el sargento Juan siguió hablando:

“Alguien podría pensar que el resultado exitoso de un ejercicio como este sería que el cabo Carlos hubiera entrado con una enorme cabellera y que se volviera pelado en un abrir y cerrar de ojos ante ustedes.”

Una risa incómoda recorrió el lugar.

“Creanlo o no, algo así es lo que viví desde mi perspectiva. Cuando entramos con el cabo Carlos, él realmente tenía mucho más pelo que ahora. Yo viajé al pasado y lo rapé, pero desde la perspectiva de ustedes, él estuvo rapado todo el tiempo desde que ingresó porque lo rapé antes de que entráramos. A la realidad que se genera luego de un viaje en el tiempo, la llamamos iteración. Si antes de mi viaje estábamos viviendo en la iteración N, ahora estamos en la iteración N + 1”.

Yo debí haber puesto la misma cara de desconcierto que el conductor.

La disconformidad del público aumentó y lo hizo saber.

“Tranquilidad, por favor”, gritó el conductor. “Tengo entendido que los militares va a realizar una segunda demostración”.

“Así es”, dijo Carlos, tomando la palabra por primera vez en la noche. “Esta segunda prueba tendrá un impacto visual y calmará un poco las ansias. Voy a meter esta lapicera en esta caja y la voy a colocar aquí en el centro. Ahora me voy a dar vuelta por veinte segundos. Luego voy a viajar al pasado a robarme la lapicera y traerla de regreso”. Cerró la caja, se dió vuelta y empezó a contar.

“Uno, dos, tres…”

“¡Holly shit!” gritó el conductor.

“¡La puta que lo parió”, grité yo.

El sargento Juan sonreía.

El cabo Carlos seguía contando tranquilo de espaldas: “...cinco, seis…”.

Y otro Carlos entró al estudio, frente a todas las cámaras y todo el público, en puntas de pie. Todo el mundo gritaba y este se llevaba un dedo a los labios para pedirles silencio. Parecía un mimo y con sus gestos se burlaba de su otro yo que contaba de espaldas. Luego, con la misma suavidad con la que se movía, destapó la caja, sacó la lapicera y la volvió a cerrar. Con exagerados gestos se la llevó al bolsillo de la camisa y luego, nuevamente con una sobreactuada forma de caminar en puntas de pie, salió tras bambalinas.

El primer cabo Carlos terminó de contar y se dio vuelta. “¡Ya vuelvo!”, gritó y tocó algo en su muñeca. Desapareció y a los veinte segundos exactos volvió a aparecer en el mismo lugar. Se llevó la mano al bolsillo de la camisa y mostró la lapicera. No hizo falta que vuelva a abrir la caja. Habíamos entendido.

Un grupo de personas irrumpió con violencia intentando robarle los relojes a los uniformados. Yo miraba la pantalla de pie. Entonces más uniformados entraron en escena y luego la transmisión se cortó. Caí sentado en mi sillón, tratando de procesar lo que había visto. Y recordando, recordando una promesa que me había hecho muchos años atrás.

Al día siguiente los medios daban cuenta de lo que había sido la transmisión. En la redacción, podía no diferir mucho de otros experimentos o experiencias o ejercicios (el nombre es lo de menos) que habían sido descritos antes. Pero el efecto era distinto. Ahora era algo un poco más real. Ya no era teoría.

De aquella primera demostración pública pasaron ya casi cuatro años. El resto es historia conocida: celebrities experimentando viajes y contando su experiencia a todo el mundo a través de las pantallas, anuncios del gobierno, las leyes que se proclamaron y luego lo inevitable: las manifestaciones. En todas las capitales del mundo y en ciudades más chicas también, las personas se concentraban exigiendo a sus gobernantes que esa tecnología estuviera disponible. Un bien más para el consumo.

Los primeros fueron Estados Unidos y Alemania. Acá solo nos llegaban las publicidades. Me acuerdo de una particularmente famosa en la que se mostraba a un hombre de unos treinta y cinco años corriendo por un campo en China. El mismo hombre tomaba un vuelo, después corría por un aeropuerto y tomaba otro, después un tren, después un bus, después un taxi, siempre con la misma música punzante de fondo y el tic-tac de un reloj apenas perceptible. La publicidad terminaba cuando el hombre entraba corriendo a un hospital, abriendo con fuerza puertas blancas que se abrían al medio, una, dos, tres, hasta llegar a una cama vacía. La imagen se fundía a blanco y un mensaje en letras azules era proyectado: “Dé ese último abrazo”.

Con el tiempo, los videos de las celebrities se hundieron en las pantallas y en su lugar flotaron más alto videos publicados por personas comunes contando sus historias, mostrando sus pequeños souvenirs, inofensivos objetos, como un autito de colección cuya obtención había sido autorizada. Lo más común era despedirse de familiares que habían muerto, pero también había historias más raras, como la de ese que usó sus cinco minutos en viajar al 22 de junio del año 1986, Argentina-Inglaterra, minuto 55 para ver cuando Maradona convertía el mejor gol de la historia. Por supuesto, cuando alguien lograba que le autoricen una originalidad como esa, muchas otras personas lo imitaban y cuando se llenaba de gente, ya nadie podía volver ahí.

Desde mañana se van a poder tramitar solicitudes de viaje en el tiempo.



Solo un mes después del anuncio pude acceder al sistema para solicitar mi viaje. Antes era imposible, todo estaba saturado. Sospecho que se trató de una saturación artificial. Un mecanismo impulsado por los poderosos para poder viajar primero. Una especie de primera clase. Pasa en los recitales, en los partidos de fútbol. ¿Por qué no iba a pasar con esto? Un mes entero esperé. Solo después de ese tiempo el sistema empezó a responder, a reaccionar, y los que no teníamos contactos, los caminantes, los que no nos acomodábamos, pudimos acceder.


\chapter*{Yumiko} \addcontentsline{toc}{chapter}{Yumiko}


Yumiko nació en Kodaira, al oeste de Tokio. Su mamá servía mesas en un gran comedor, junta a la planta principal de una compañía multinacional de neumáticos. Así, comentandole cual era el menú del día, la mamá de Yumiko conoció a quién sería su esposo.

Yumiko creció sin hermanos y realizó en la ciudad sus estudios primarios y secundarios. A la hora de elegir una universidad, optó por estudiar medicina en la Universidad de Tokio.

Un año antes de obtener su título, conoció a un joven que estudiaba literatura en la misma universidad. El joven la invitó a viajar como mochileros por sudamérica y Yumiko, desoyendo el consejo de sus padres, aceptó maravillada.

Su vuelo los dejó en Río de Janeiro y luego de pasar pocos días en la ciudad volaron a Cuzco, donde realmente empezaba su viaje. Caminaban durante el día y acampaban en la noche. Yumiko no hablaba español, pero su compañero sí y se encargaba de hacer las compras de provisiones o de boletos para cuando querían recorrer distancias mayores a las que su ritual de caminar-acampar les permitía.

El joven describía la naturaleza en sus versos mientras que Yumiko registraba en su cuaderno toda la información que obtenía de los pobladores locales sobre plantas medicinales.

En las calles de Potosí una mujer con ropas coloridas pero el rostro cobrizo vendía un extraño ungüento a base a una planta del lugar que, según decía, sanaba cualquier dolor de los huesos. Yumiko, a través de su compañero-traductor, le insistió en que le revele el nombre de aquella planta pero la mujer, cuyas palabras eran apenas audibles, era un muro infranqueable para las acotadas palabras que el muchacho manejaba.

Habían pasado tres meses de su llegada al continente cuando abandonaron Bolivia para adentrarse en Chile. Recorrieron la cordillera durmiendo en posadas de pequeños pueblos ya que las bajas temperaturas ponían en jaque a su carpa. Los pocos ahorros que habían llevado eran reforzados ofreciendo los servicios de enfermería de Yumiko y en una de sus escalas obtuvieron una buena cantidad de dinero pintando por completo una casa.

Una mañana en que el sol rompía en haces púrpuras contra un lago, cruzaron la frontera con el objetivo de empezar a subir en lo que sería la segunda parte de su viaje.

Habían comprado una bicicleta doble y con ella recorrían una ruta hacia el norte, cuando un camión los embistió en lo cerrado de la noche.

Yumiko fue la única que sobrevivió al accidente. Su compañero y el camionero cerraron los ojos para siempre esa noche pero ella los volvió a abrir catorce meses después.

Cuando despertó no recordaba nada. A eso se sumaba el problema de que no conocía el idioma. Los datos que se pudieron recuperar del accidente eran mínimos: su nombre y ciudad de origen en una credencial de la universidad. También se encontró un pasaje de vuelta a Japón que ya había vencido cuando ella despertó. Hubo algunos escasos y burocráticos intentos diplomáticos por devolverla a su país pero nunca se concretaron. Se logró una comunicación con sus padres, pero eran ancianos y no tenían dinero como para buscarla o repatriarla.

Cuando estuvo lista para abandonar el hospital, sus autoridades se encontraban en la disyuntiva de qué hacer con ella. Entonces, la familia del camionero se presentó y se ofreció a recibirla en su casa mientras continuaba con sus rehabilitación y recuperaba la memoria.

Así fue que durante el siguiente año, Yumiko no solo volvió a caminar y recuperó poco a poco la memoria, sino que también aprendió a hablar español.

En la casa vivía la madre y la hermana del camionero. Ahora eran tres mujeres que se acompañaban en el día a día.

Cuando Yumiko estuvo sana y fuerte, llamó a sus padres. Les dijo que no iba a volver. Quería devolver con su vocación algo a esa tierra que le había salvado la vida.

En cada aniversario del día que Yumiko abandonó el hospital, las tres mujeres hacían un pequeño festejo, como si de su cumpleaños se tratara y Yumiko soplaba una vela sobre una pequeña torta.

La madre y la hermana trabajaban cosiendo y lavando ropa. Durante dos años Yumiko hizo lo mismo que ellas durante el día. Durante la noche cursaba la carrera de enfermería. Cuando consiguió su título oficial, se presentó en el mismo hospital donde había permanecido en coma durante catorce meses y entregó su solicitud de trabajo en la recepción.

Trabajó en la institución hasta que se jubiló. Y cuando lo hizo, siguió trabajando como cuidadora de ancianos.

Una noche, el viejo de los microondas fue internado en el lugar. Yumiko fue designada para cuidarlo. En otra iteración de esta historia, el viejo le dijo lo que sentía por ella.


\chapter*{Historia del aprendiz de programador} \addcontentsline{toc}{chapter}{Historia del aprendiz de programador}


Juan Andrés llegó desde el pueblo con la idea romántica de convertirse en un programador de Software Libre. En la ciudad podría adquirir todos los conocimientos necesarios (iría a la universidad, asistiría a charlas y talleres) pero también podría ponerse en contacto con programadores de experiencia: auténticos hackers que trabajaban día a día mejorando software, manteniendo redes o configurando toda clase de equipamiento.

Atrás quedaba el pequeño grupo de usuarios en el que a fuerza de leer revistas extranjeras exprimían el poco conocimiento al que podían acceder. En la ciudad también tendría la posibilidad de conectarse a Internet siempre que quisiera y, con eso, más conocimiento del que podría absorber.

El primer día que llegó a la universidad recibió una pequeña hoja de papel con sus horarios y el carnet de la biblioteca. Guardó el carnet en la billetera y miró el cronograma. Eran las nueve de la mañana y su primera clase, Algoritmos y estructuras de datos, había empezado hacía cuarenta y cinco minutos. (Por un momento se arrepintió de haber elegido el turno mañana en lugar del turno tarde). Sin perder tiempo subió las escaleras y corrió por un largo pasillo mientras iba pasando por aulas numeradas: 12, 13, 14, 15. Esa era, aula 15. Cuando entró, ya todos los demás estaban sentados en sus lugares y un profesor canoso hablaba con una tiza en la mano.

---Llega tarde, alumno ---le dijo por todo saludo y apuntado con la tiza le indicó, cerca del fondo, el único banco libre del salón.

El profesor, que se apellidaba Marina, continuó hablando y Juan Andrés alcanzó a anotar en su cuaderno la conclusión de aquella disertación: mientras más poderosas las estructuras de datos, más simples los algoritmos que tendrían que escribir. El corolario lógico de eso era que ante un problema, uno no debía, como primer paso, saltar a escribir código a toda velocidad sino tomarse un tiempo para evaluarlo y pensar cuál sería la mejor estructura de datos para resolverlo. Una vez que se había decidido eso, el algoritmo para terminar de darle solución al problema se revelaba solo, ante el programador, de una forma casi mágica.

---¿Qué lenguaje vamos a usar en esta clase? ---le preguntó Juan Andrés a Gonzalo, el muchacho que estaba sentado junto a él.

---C. Pero yo prefiero este ---dijo el otro. Y sacó de su mochila un ejemplar original de Programming Pearl, con el famoso dibujo del dromedario en la tapa.

Juan Andrés tomó nota mental de encargar una copia de ese libro en la fotocopiadora más tarde.

---Es un poco complicado ---siguió Gonzalo---, pero mucho más dinámico. Y mirá esto.

Mientras Marina seguía hablando, Gonzalo abrió el libro en una de las primeras páginas y señaló: estas son, según Larry Walls, las tres virtudes de los programadores: pereza, impaciencia e hibris. Pereza: cuando tenés que hacer algo por una segunda vez, antes de hacerlo preferís escribir un programa que lo haga por vos; esa y todas las veces siguientes. Es lo que hace que escribas programas que ahorren trabajo y escribas documentación para no tener que responder una y otra vez la misma pregunta. Impaciencia: cuando ese programa que escribiste corre demasiado lento, lo optimizás para que corra más rápido.

---¿Y qué es Hibris?

---Orgullo excesivo o ego desmesurado. Es una palabra griega. Es lo que hace que quieras publicar programas en internet para que otros vean tu código. Programas tan bien escritos que nadie pueda quejarse de ellos. ¡Aunque lograr esto último es imposible! ¡Jaja!

---En resumen… ---dijo Juan Andrés que se había quedado un poco perdido.

---Que los programadores somos haraganes, impacientes y arrogantes.

---¿Somos? ¿Vos ya sabés programar?

Gonzalo se rió.

---Desde hace unos cinco años. Trabajo programando en la empresa de mi papá.

Le dijo su nombre y le tendió la mano.

Volvieron a darse la mano cuando un par de meses después decidieron formar equipo para resolver el trabajo práctico de la materia. La propuesta del profesor era desarrollar una hoja de cálculo desde cero; tendrían que elegir una estructura de datos a utilizar (¿una matriz?, ¿listas doblemente enlazadas?, ¿árboles binarios?) y escribir los algoritmos necesarios para que un usuario pueda realizar las operaciones de guardar un valor o una fórmula en una celda y leer el valor de una celda. La dificultad recaía en que una fórmula podía hacer referencia a otras celdas de la hoja de cálculo y los algoritmos de lectura debería tener esto en cuenta a la hora de leer un valor (¿calcularían cada valor cuando sea requerido o irían guardado los valores precalculados en alguna estructura de datos intermedia?). Muchas eran las preguntas y poco el tiempo para responderlas: tres semanas después debían entregar el trabajo y defenderlo.

Juan Andrés no había mudado su computadora desde el pueblo así que todos los días, después del cursado, se juntaban en la casa de Gonzalo a trabajar. Gonzalo tenía su computadora en la pieza, así que podían quedarse trabajando hasta tarde, sin ser interrumpidos. Todos los días, a eso de las diez de la noche, la madre de Gonzalo tocaba la puerta y entraba con una bandeja con sándwiches o alguna otra cosa que se pueda comer rápido. Este sistema les funcionó durante las primeras dos semanas, pero en la tercera notaron que el tiempo se les iba acabando y aún no habían completado las funcionalidades requeridas, mucho menos habían podido probar en forma exhaustiva el programa, por lo que decidieron seguir trabajando en las computadoras de la facultad durante las horas libres y durante las horas de materias más aburridas.

La facultad contaba con un laboratorio de diez computadoras del cual los alumnos podían disponer libremente previa obtención de un turno. Por lo general no había mucha demanda y siempre que llegaban a la sala solo tenían que firmar una planilla y sentarse en alguna de las computadoras desocupadas. Insertaban con mucho cuidado el disquete que tenía la última versión de su trabajo y se ponían a trabajar. Tipeaba uno u el otro en forma indistinta, su estilo de programación era similar por lo que no caían en las todas peleas de los programadores por el formato del código y los dos tenían conocían de principio a fin el programa. Eso sería útil para la defensa, ya que el profesor podría preguntar a cualquiera de los dos por cualquier parte del código. Cuando tenían que irse, se mandaban una copia del trabajo por mail y pisaban la versión del disquete; así y todo, a veces podían encontrarse nombres de archivo conviviendo como: VERSION-FINAL.zip, VERSION-FINAL-2.zip, VERSION-FINAL-3.zip, VERSION-FINAL-ULTIMA.zip, VERSION-FINAL (esta si es la versión final).zip.

El jueves, luego de clases, se fueron a la casa de Gonzalo a dar los últimos retoques a su trabajo. Si bien solo faltaban algunos detalles de la interfaz de usuario, se quedaron despiertos toda la noche, haciendo pequeños ajustes y modificaciones. “¿Qué tal si le ponemos un menú desplegable?”, “Podríamos permitir pintar celdas de colores”, “¡Vamos a hacer que cuando el usuario apriete tal combinación de teclas se ejecute tal funcionalidad!”. La noche pasó entre Coca-Cola y cambios y se cumplió la Ley neumática de la programación. Esta ley dice que el trabajo que tiene que hacer un programador se comporta igual que un gas en un recipiente; donde el recipiente es el tiempo que el programador tiene para completar la tarea. El trabajo se expandirá para ocupar todo el tiempo.

El viernes a la mañana, si no hubiese sido por la madre de Gonzalo, se hubiesen quedado dormidos. Cuando los despertó, faltaban veinte minutos para el horario de la entrega. Corrieron con el mentado disquete en la mano y justo cuando llegaban a la esquina también lo hacía el colectivo que tenían que tomar. Luego de quince minutos eternos, se bajaron en la puerta de la facultad y llegaron corriendo al aula 15. Cuando entraron estaba vacía. ¿Ya se habían ido todos? Deshicieron sus pasos y en la puerta encontraron un papel pegado: “Nos trasladamos al aula 21”. Eso quedaba en el otro ala de la facultad así que volvieron a correr. Cuando estaban llegando, ahora sí, se encontraron con otros alumnos. Ya todos habían entregado sus trabajos y debían esperar a que sean revisados para luego defenderlos. Un jefe de trabajos prácticos los vió llegar y les pidió su entrega. Le entregaron el disquete y una copia impresa de su trabajo. Ahora solo había que esperar.

Aprovecharon el tiempo de espera para charlar con algunos compañeros. Hasta ese día, todos habían sido muy celosos de su implementación y no habían compartido con miembros de otros grupos las decisiones que habían tomado, pero ahora la suerte estaba echada, todos habían alcanzado, de alguna forma, el objetivo planteado (si no no estarían ahí) y el clima relajado que eso generaba, propiciaba el intercambio. No sabían cuánto tiempo habían esperado, cuando el jefe de trabajos prácticos los hizo pasar.

Marina los saludó por su apellido y, con un gesto difícil de descifrar, se sacó los anteojos y los volvió a mirar.

---¿Qué vamos a hacer con ustedes?

Los compañeros no sabían a qué se refería.

---Tengo dos trabajos exactamente iguales. Uno es el de ustedes y el otro es de un grupo del turno tarde.

---¡¿QUÉ?! ---Gonzalo saltó de su silla.

---No puede ser, no puede ser ---se repetía Juan Andrés más para sí mismo que para los otros---. Tiene que haber un error.

---No hay error. Se lo puedo mostrar si quieren: los nombres de las variables son distintos y algunas funciones están cambiadas de lugar, pero es obvio que uno es una copia del otro. Al menos que…

---¿Al menos que qué? ---se apresuró a preguntar Juan Andrés.

---A menos que no hayan cumplido la consigna y lo hayan hecho entre cuatro.

---No, nos lo tienen que haber copiado ---dijo Gonzalo sin un atisbo de duda en su voz.

---Los del otro grupo dicen lo mismo ---concluyó el profesor---. Así que vamos a hacer la defensa en conjunto. Confío en que los autores originales van a poder explicar el programa mucho mejor que los ladrones.

---¡Eso seguro! ---Gonzalo habló confiado.

---Muy bien ---dijo por fin el profesor---. Salgan un momento y luego vamos a llamar a ambos grupos.

Los compañeros abandonaron el aula con un sabor agrio en la boca. ¿Cómo podía haber pasado eso? ¿Podría alguien haberse metido en la computadora de Gonzalo para robarles el trabajo? No, eso era ridículo. Alguien con esas habilidades podría haber resuelto sin esfuerzo el trabajo. Tenía que ser otra cosa. Entonces, al mismo tiempo, ambos llegaron a la misma conclusión: las computadoras públicas. No estaban seguros de, todas las veces, haber borrado el trabajo luego de copiarlo al disquete.


\cleardoublepage

%IMPRENTA
%\includepdf{empty.pdf}

\hspace{0pt}
\vfill
\begin{center}
Maqueteado automáticamente utilizando \emph{Automágica}.
\bigbreak
http://www.juanjoconti.com/automagica/
\end{center}
\vfill
\hspace{0pt}
\end{document}